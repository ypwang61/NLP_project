\documentclass[5pt]{beamer}
\usetheme{AnnArbor}
\usecolortheme{seahorse}
\setbeamerfont{block title}{size=\small}
\setbeamerfont{block body}{size=\scriptsize}
\begin{document}
\begin{frame}
\frametitle{Milestones in World History}
\begin{columns}
\begin{column}{0.3\textwidth}
\begin{block}{\textbf{Birth of Democracy in Ancient Greece}}
The concept of democracy was first conceived in Athens around the 5th century BC. It allowed the citizens of Athens to participate directly in political decision-making, revolutionizing future governance systems.
\end{block}
\end{column}
\begin{column}{0.3\textwidth}
\begin{block}{\textbf{Renaissance Period and its Impact}}
The Renaissance Period, from the 14th to the 17th century, was a major cultural movement in Europe. It marked the transition from the Middle Ages to Modernity and revived interest in Greco-Roman philosophy, art, and science.
\end{block}
\end{column}
\begin{column}{0.3\textwidth}
\begin{block}{\textbf{The Two World Wars and Their Aftermath}}
The occurrence of the two World Wars in the first half of the 20th century significantly changed global politics and society. Post these wars, numerous countries gained independence from colonial powers, and international bodies like the United Nations were established.
\end{block}
\end{column}
\end{columns}
\end{frame}
\end{document}
